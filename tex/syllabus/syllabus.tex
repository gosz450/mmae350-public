\documentclass[12pt]{article}
\usepackage[margin=1in]{geometry}
\usepackage{longtable}
\usepackage{array}
\usepackage{hyperref}

\begin{document}

\begin{center}
    \Large \textbf{MMAE 350: Computational Mechanics} \\
    \large Spring 2026 \\
    \vspace{0.5em}
    Department of Mechanical, Materials, and Aerospace Engineering \\
    Illinois Institute of Technology
\end{center}

\vspace{1em}

\section*{Instructor Information}
\begin{itemize}
    \item \textbf{Instructor:} Dr. Michael Gosz
    \item \textbf{Office:} 207A Retaliata Engineering Center    \item \textbf{Phone:} (312) 567-3198
    \item \textbf{Email:} gosz@illinoistech.edu
    \item \textbf{Office Hours:} TBA
    \item \textbf{Class Meetings:} Tue, Thur (11:25am -12:40pm) Stuart Building 113
\end{itemize}

\section*{Prerequisites}
MATH 251 (Multivariate and Vector Calculus), MATH 252 (Intro to Differential Equations), MMAE 202 (Mechanics of Solids II).  
Programming background recommended.

\section*{Textbook}
M. Gosz, \textit{Computational Mechanics: A Modern Introduction with Machine Learning and AWS Workflows}, 1st edition.  
Other resources and handouts will be posted on the course Canvas site.

\section*{Course Description}
This course introduces computational methods used in mechanical and
aerospace engineering, with an emphasis on numerical solution of
engineering governing equations. Students develop computational fluency
using Python and Jupyter notebooks, apply finite-difference,
finite-volume, and introductory finite-element methods, and connect
numerical algorithms to physical conservation laws in heat transfer,
wave propagation, and transport phenomena.

\section*{Learning Objectives}

Upon successful completion of this course, students will be able to:
\begin{enumerate}
  \item Develop and execute engineering computations using Python, NumPy,
        SymPy, and Jupyter notebooks.
  \item Formulate and solve systems of linear algebraic equations arising from
        discretized engineering problems.
  \item Apply numerical root-finding techniques, including Newton’s method,
        to nonlinear equations and systems encountered in engineering analysis.
  \item Derive finite-difference approximations for spatial and temporal
        derivatives and implement them for transient and steady-state problems.
  \item Analyze the stability and accuracy of explicit and implicit numerical
        schemes for time-dependent partial differential equations.
  \item Solve one- and two-dimensional heat conduction problems using
        finite-difference methods and matrix-based solution strategies.
  \item Model wave propagation in elastic solids using explicit
        time-marching finite-difference schemes.
  \item Apply finite-volume concepts to conservation laws and implement
        numerical solutions of advection--diffusion equations.
  \item Distinguish between strong and weak formulations of governing equations
        and construct simple finite element approximations for one-dimensional
        problems.
  \item Interpret numerical results in the context of physical behavior,
        assess solution quality, and communicate computational findings clearly.
\end{enumerate}

\section*{Topics Covered}
\begin{enumerate}
  \item Introduction to computational mechanics and engineering workflows
        using Python and Jupyter notebooks
  \item Matrix and vector algebra for engineering applications
  \item Numerical solution of nonlinear equations and systems using
        Newton’s method
  \item Direct and iterative methods for solving systems of linear
        algebraic equations
  \item Governing equations of heat transfer, wave propagation, and
        transport phenomena
  \item Finite-difference methods for steady and transient partial
        differential equations
  \item Stability and accuracy analysis of time-marching numerical schemes
  \item Finite-volume methods for conservation laws and advection--diffusion
        problems
  \item Introduction to weak formulations and one-dimensional finite
        element methods
\end{enumerate}

\section*{Computer Usage}
Students will use Python Jupyter Notebooks to complete assignments.

\section*{Grading Policy}

Student performance in this course will be evaluated using the following
components:

\begin{itemize}
  \item \textbf{Homework Assignments (30\%)} \\
  Weekly or biweekly problem sets focused on analytical derivations,
  numerical methods, and short computational exercises.

  \item \textbf{Computational Notebooks and Labs (25\%)} \\
  Jupyter notebook assignments emphasizing implementation of numerical
  algorithms, interpretation of results, and clear documentation of
  computational workflows.

  \item \textbf{Midterm Exams (20\% total; 10\% each)} \\
  Two in-class midterm exams, equally spaced throughout the semester,
  assessing conceptual understanding, mathematical formulation, and
  problem-solving skills for numerical methods covered up to each exam.

  \item \textbf{Final Exam or Final Project (25\%)} \\
  A comprehensive assessment or project evaluating students’ ability to
  integrate numerical methods, physical modeling, and computational
  implementation for an engineering problem.
\end{itemize}
\pagebreak
% --------------------------------------------------
\section*{Course Calendar — Spring 2026}
\renewcommand{\arraystretch}{1.2}
\begin{longtable}{|>{\raggedright\arraybackslash}p{2.5cm}|>{\raggedright\arraybackslash}p{1.5cm}|>{\raggedright\arraybackslash}p{9cm}|}
\hline
\textbf{Date} & \textbf{Day} & \textbf{Topic} \\ \hline
\endfirsthead
\hline
\textbf{Date} & \textbf{Day} & \textbf{Topic} \\ \hline
\endhead

Jan 13 & Tue &
Course overview; role of computation in engineering; Python, Jupyter, and course workflow \\

Jan 15 & Thu &
Arrays, vectors, and matrices in Python; numerical vs.\ symbolic computation \\

Jan 20 & Tue &
Matrix algebra for engineering systems; solving linear systems \\ \hline
%================================================
Jan 22 & Thu &
Direct methods for linear systems: Gaussian elimination and LU decomposition \\

Jan 27 & Tue &
Matrix structure and sparsity; tridiagonal systems and the Thomas algorithm \\

Jan 29 & Thu &
Iterative solvers for linear systems; Gauss--Seidel method \\ \hline
%=================================================
Feb 03 & Tue &
Nonlinear equations of one variable; Newton’s method \\

Feb 05 & Thu &
Newton’s method for systems of equations; engineering examples \\

Feb 10 & Tue &
Applications of nonlinear systems; conditioning and convergence \\ \hline
%=========================================
Feb 12 & Thu &
Review and problem-solving session (Chapters 1--2) \\

Feb 17 & Tue &
Exam~1 \\ \hline

Feb 19 & Thu &
Derivation of the heat equation; boundary and initial conditions \\ 

Feb 24 & Tue &
Finite-difference discretization of the 1D heat equation; FTCS method \\

Feb 26 & Thu &
Stability analysis; CFL condition; von Neumann analysis \\

Mar 03 & Tue &
Implicit time integration; Crank--Nicolson method \\

Mar 05 & Thu &
Matrix formulation of transient heat conduction problems \\ \hline

Mar 10 & Tue &
Two-dimensional heat conduction; five-point stencil \\

Mar 12 & Thu &
Steady vs.\ transient 2D heat conduction; computational examples \\ \hline

Mar 17 & Tue & SPRING BREAK — No Class \\
Mar 19 & Thu & SPRING BREAK — No Class \\ \hline

Mar 24 & Tue &
Wave propagation in elastic solids; 1D wave equation \\

Mar 26 & Thu &
Explicit time-marching schemes; stability and the CFL condition \\ \hline

Mar 31 & Tue &
Conservation laws and transport phenomena; integral vs.\ differential forms \\

Apr 02 & Thu &
Finite-volume discretization of the advection--diffusion equation \\

Apr 07 & Tue &
Upwind vs.\ central difference schemes; numerical diffusion \\ \hline

Apr 09 & Thu &
Weak formulations; introduction to finite element concepts \\

Apr 14 & Tue &
One-dimensional finite element method; shape functions and weak form \\

Apr 16 & Thu &
Assembly of the global system; solution and interpretation \\ \hline

Apr 21 & Tue &
Review for Exam~2 \\

Apr 23 & Thu &
Exam~2 \\

Apr 28 & Tue &
Computational project workshop; verification and validation \\

Apr 30 & Thu &
Course synthesis; comparison of FD, FV, and FEM methods \\ \hline

May 05 & Tue &
Final Exam or Project Presentations (Registrar-scheduled slot) \\

May 07 & Thu &
Final Exam or Project Presentations (Registrar-scheduled slot) \\ \hline
\end{longtable}

% --------------------------------------------------
\section*{Course Policies}

\subsection*{Attendance}
Regular attendance is expected. Students are responsible for all material covered in class and all announcements made, regardless of attendance. Excessive unexcused absences may negatively impact your performance and grade.

\subsection*{Late Work}
Assignments are due on the date and time specified. Late work will generally not be accepted unless prior arrangements are made with the instructor, or unless extraordinary circumstances can be documented. In such cases, partial credit may be given at the discretion of the instructor.

\subsection*{Academic Integrity}
Illinois Tech expects all students to uphold the highest standards of academic honesty. Cheating, plagiarism, or any other form of academic dishonesty will not be tolerated. Violations will be reported and may result in failure of the assignment, failure of the course, and/or additional disciplinary action as outlined in the Illinois Tech Code of Conduct. For more information, see: \url{https://www.iit.edu/student-affairs/student-handbook}

\subsection*{Collaboration}
Collaboration on homework assignments is permitted at the level of discussing concepts and approaches. However, all work turned in must be your own. Copying code, solutions, or written responses from another student or from online sources constitutes academic dishonesty.

\subsection*{Use of Technology}
Students are encouraged to use Python, and other computational tools as required by the course. Any use of technology during quizzes or exams must be explicitly permitted by the instructor. Unauthorized use of technology during exams will be considered a violation of academic integrity.

\subsection*{Communication}
Course announcements will be made in class and via email or the course LMS (Canvas). It is your responsibility to check email and the course site regularly. Email is the preferred method of communication outside of class hours.

\subsection*{Professional Conduct}
Respectful behavior is expected in class, in labs, and in all course-related activities. Disruptive conduct will not be tolerated. Students should contribute to a learning environment that supports diversity of thought and experience.

% --------------------------------------------------
\section*{Accessibility Statement}

Illinois Tech is committed to providing an inclusive educational environment and making every effort to ensure equal access for all students. If you are a student with a documented disability and require reasonable academic accommodations, please contact the Center for Disability Resources (CDR) as soon as possible. Accommodations are determined on a case-by-case basis, considering the student’s documented needs and the technical requirements of the course.

To request accommodations or learn more, contact:
\begin{itemize}
  \item \textbf{Center for Disability Resources (CDR)} \\
    Phone: 312-567-5744 \\
    Email: \texttt{disabilities@illinoistech.edu}
\end{itemize}

Students must submit documentation and meet with CDR to establish eligibility and receive an accommodation letter. Provide your letter to the instructor early in the semester to discuss appropriate implementation of accommodations.

For more information, visit: \url{https://www.iit.edu/cdr}

% --------------------------------------------------
\section*{University Academic Calendar — Spring 2026}
\renewcommand{\arraystretch}{1.2}
\begin{longtable}{|>{\raggedright\arraybackslash}p{4cm}|>{\raggedright\arraybackslash}p{10cm}|}
\hline
\textbf{Date} & \textbf{Event} \\ \hline
January 12 & Spring Courses Begin \\
January 19 & Martin Luther King, Jr. Day — No Classes \\
January 20 & Last Day to Add/Drop for Full Semester Courses with No Tuition Charges \\
January 27 & Last Day to Request Late Registration \\
March 13 & Midterm Grades Due \\
March 16--21 & Spring Break Week — No Classes \\
March 30 & Last Day to Withdraw for Full Semester Courses \\
April 6 & Fall Registration Begins \\
May 2 & Last Day of Spring Courses \\
May 3 & Last Day to Request an Incomplete Grade \\
May 4--9 & Final Exam Week (Final Grading Begins May 4) \\
May 13 & Final Grades Due at Noon (CST) \\
\hline
\end{longtable}

\end{document}