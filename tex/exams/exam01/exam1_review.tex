\documentclass[11pt]{article}

\usepackage{amsmath, amssymb}
\usepackage{geometry}
\usepackage{fancyhdr}
\usepackage{enumitem}
\usepackage{bm}

\geometry{margin=1in}

\pagestyle{fancy}
\fancyhf{}
\lhead{MMAE 350}
\rhead{Exam 1}
\rfoot{\thepage}

\begin{document}

\begin{center}
{\Large \textbf{Illinois Institute of Technology}}\\[4pt]
{\Large \textbf{MMAE 350 -- Computational Mechanics}}\\[6pt]
{\Large \textbf{Exam 1}}\\[6pt]
Time: 75 minutes \\
Total: 50 points
\end{center}

\vspace{1em}

\noindent
\textbf{Name:} \rule{3in}{0.4pt}

\vspace{1em}

\noindent
\textbf{Instructions:}
\begin{itemize}
\item This exam is closed book and closed notes.
\item A calculator is permitted.
\item Show all work clearly.
\item Answers without supporting work may receive limited credit.
\end{itemize}

\vspace{1em}
\hrule
\vspace{1em}

% ==============================================================
\section*{Part I -- Gaussian Elimination (6 points)}

Consider the system

\[
\begin{aligned}
2x + y - z &= 3 \\
4x + 3y + z &= 7 \\
-2x + y + 2z &= 1
\end{aligned}
\]

\begin{enumerate}[label=(\alph*)]

\item (3 pts) Compute the multiplier $m_{21}$ used to eliminate the first column (no pivoting).

\vspace{2cm}

\item (3 pts) After eliminating the first column, what is the updated coefficient in row 2, column 2?

\vspace{2cm}

\end{enumerate}

\newpage

% ==============================================================
\section*{Part II -- Thomas Algorithm (6 points)}

Consider the tridiagonal system

\[
\begin{bmatrix}
2 & -1 & 0 \\
-1 & 2 & -1 \\
0 & -1 & 2
\end{bmatrix}
\bm{x}
=
\begin{bmatrix}
1 \\
0 \\
1
\end{bmatrix}.
\]

\begin{enumerate}[label=(\alph*)]

\item (3 pts) Perform the first forward elimination step and compute the modified diagonal entry $d_2'$.

\vspace{3cm}

\item (3 pts) In 2--3 sentences, explain why the Thomas algorithm is more efficient than standard Gaussian elimination for this type of matrix.

\vspace{3cm}

\end{enumerate}

% ==============================================================
\section*{Part III -- Gauss--Seidel Method (6 points)}

Consider the linear system

\[
\begin{bmatrix}
4 & -1 & 0 \\
-1 & 4 & -1 \\
0 & -1 & 4
\end{bmatrix}
\bm{x}
=
\begin{bmatrix}
2 \\
6 \\
2
\end{bmatrix}.
\]

\begin{enumerate}[label=(\alph*)]

\item (4 pts) Perform one Gauss--Seidel sweep starting from
\[
\bm{x}^{(0)} = (0,0,0),
\]

and compute
\[
x_1^{(1)}, \quad x_2^{(1)}, \quad x_3^{(1)}.
\]

\vspace{4cm}

\item (2 pts) In 2--3 sentences, explain why this system is expected to converge under Gauss--Seidel iteration.

\vspace{3cm}

\end{enumerate}

\newpage

% ==============================================================
\section*{Part IV -- Newton's Method (1D) (6 points)}

Let
\[
f(x) = x^3 - 2x - 5.
\]

\begin{enumerate}[label=(\alph*)]

\item (2 pts) Compute $f'(x)$.

\vspace{2cm}

\item (2 pts) Perform one Newton iteration starting from $x_0 = 2$. Compute $x_1$.

\vspace{3cm}

\item (2 pts) At iteration $x_k$, Newton’s method replaces $f(x)$ with a linear approximation. In 2--3 sentences, explain how $x_{k+1}$ is obtained from this linearization.

\vspace{3cm}

\end{enumerate}

% ==============================================================
\section*{Part V -- Newton's Method for a System (8 points)}

Consider
\[
f(x,y) = x^2 + y^2 - 4,
\]
\[
g(x,y) = x - y - 1.
\]

with initial guess $(1,1)$.

\begin{enumerate}[label=(\alph*)]

\item (2 pts) Compute the residual vector at $(1,1)$.

\vspace{3cm}

\item (4 pts) Compute the Jacobian matrix at $(1,1)$.

\vspace{3cm}

\item (2 pts) In 2--3 sentences, explain why solving the linear system inside Newton's method corresponds to a first-order Taylor approximation.

\vspace{3cm}

\end{enumerate}

\newpage

% ==============================================================
\section*{Part VI -- Big Picture (6 points)}

In 4--6 complete sentences, compare the following three methods:

\begin{itemize}
\item Gaussian elimination
\item Gauss--Seidel
\item Newton's method
\end{itemize}

In your comparison, clearly address:

\begin{itemize}
\item Which types of problems each method is used to solve (linear or nonlinear),
\item Whether the method is direct or iterative,
\item One situation where each method would be appropriate.
\end{itemize}

\vspace{5cm}

% ==============================================================
\section*{Part VII -- Code Interpretation (4 points)}

Consider the following Python code:

\begin{verbatim}
x = 0
for k in range(3):
    x = 2*x + 1
print(x)
\end{verbatim}

\begin{enumerate}[label=(\alph*)]

\item (3 pts) What value is printed?

\vspace{2cm}

\item (1 pt) In one sentence, describe what this loop is doing.

\vspace{2cm}

\end{enumerate}

% ==============================================================
\section*{Part VIII -- Pseudocode (4 points)}

Write clear pseudocode for Newton's method to solve
\[
f(x) = 0.
\]


\vspace{4cm}

% ==============================================================
\section*{Part IX -- Matrix Algebra (4 points)}

Consider

\[
A =
\begin{bmatrix}
2 & -1 & 0 \\
1 & 3 & -2 \\
0 & -1 & 4
\end{bmatrix},
\quad
\bm{x} =
\begin{bmatrix}
1 \\
2 \\
-1
\end{bmatrix}.
\]

\begin{enumerate}[label=(\alph*)]

\item (2 pts) Compute $\bm{b} = A\bm{x}$.

\vspace{3cm}

\item (2 pts) Write a short block of Python code (using basic loops, not built-in matrix multiplication) that computes $\bm{b} = A\bm{x}$ for a general $3 \times 3$ matrix $A$ and vector $\bm{x}$.

\vspace{3cm}

\end{enumerate}

\end{document}