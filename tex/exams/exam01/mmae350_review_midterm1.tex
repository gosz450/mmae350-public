\documentclass[11pt]{article}

\usepackage{amsmath, amssymb}
\usepackage{geometry}
\usepackage{enumitem}
\usepackage{hyperref}

\geometry{margin=1in}

\title{MMAE 350 -- Midterm 1 Study Guide}
\author{}
\date{}

\begin{document}
\maketitle

\section*{Scope of the Exam}
This study guide is intended to help you organize your preparation for Midterm~1 and to highlight the types of concepts and skills you should be comfortable with. It is not a practice exam, and the questions listed here are meant to guide your studying rather than cover every possible topic. Midterm~1 covers material from Chapters~1 and~2 of the course text. The emphasis is on understanding fundamental numerical ideas, performing small hand calculations, and translating algorithms into clear pseudocode.
\bigskip

%------------------------------------------------
\section{Python and Computational Foundations}

\subsection*{Open-Ended Questions}
\begin{enumerate}
\item What is a virtual environment, and why is it good practice to use one?
\item What role do Jupyter notebooks play in computational mechanics?
\item Explain the difference between a Python list and a NumPy array.
\item Why are loops and conditional statements essential for numerical algorithms?
\end{enumerate}

\subsection*{Hand Calculations}
\begin{enumerate}
\item Given the vector
\[
x = [1,\, 3,\, 5,\, 7],
\]
write out explicitly what is returned by:
\[
x[1:3], \quad x[::2].
\]

\item Given the matrix
\[
A =
\begin{bmatrix}
1 & 2 \\
3 & 4
\end{bmatrix},
\]
and vector
\[
b =
\begin{bmatrix}
5 \\
6
\end{bmatrix},
\]
write out the matrix--vector product $Ax$ symbolically.
\end{enumerate}

\subsection*{Pseudocode}
\begin{enumerate}
\item Write pseudocode for a \texttt{for} loop that computes the squares of the integers from 1 to $n$.
\item Write pseudocode for an \texttt{if--else} block that checks whether a number is positive, negative, or zero.
\end{enumerate}

%------------------------------------------------
\section{Matrix Algebra}

\subsection*{Open-Ended Questions}
\begin{enumerate}
\item What does it mean for a matrix to be invertible?
\item Why is explicitly computing a matrix inverse usually discouraged in numerical work?
\item Explain the geometric meaning of the transpose of a matrix.
\item What is the physical meaning of a matrix--vector product in mechanics?
\end{enumerate}

\subsection*{Hand Calculations}
\begin{enumerate}
\item Compute the transpose of
\[
A =
\begin{bmatrix}
2 & -1 & 0 \\
1 & 3 & 4
\end{bmatrix}.
\]

\item Compute the determinant and inverse (by hand) of
\[
A =
\begin{bmatrix}
4 & -1 \\
2 & 3
\end{bmatrix}.
\]

\item Verify that $AA^{-1} = I$ for your result above.
\end{enumerate}

\subsection*{Pseudocode}
\begin{enumerate}
\item Write pseudocode for multiplying a matrix $A$ by a vector $x$ using nested loops.
\item Write pseudocode that counts how many entries of a vector are positive.
\end{enumerate}

%------------------------------------------------
\section{Solving Linear Systems \texorpdfstring{$Ax=b$}{Ax=b}}

\subsection*{Open-Ended Questions}
\begin{enumerate}
\item What is the difference between a direct method and an iterative method?
\item When is Gaussian elimination preferred over Gauss--Seidel?
\item Why does matrix structure (dense vs.\ tridiagonal) matter computationally?
\end{enumerate}

\subsection*{Hand Calculations}
\begin{enumerate}
\item Solve the system using Gaussian elimination:
\[
\begin{aligned}
2x_1 - x_2 &= 1, \\
-x_1 + 2x_2 - x_3 &= 0, \\
-x_2 + 2x_3 &= 1.
\end{aligned}
\]

\item Perform \emph{one full Gauss--Seidel sweep} starting from $x^{(0)} = (0,0,0)^T$ for the system:
\[
\begin{aligned}
10x_1 + 2x_2 + x_3 &= 13, \\
2x_1 + 10x_2 + 3x_3 &= 14, \\
x_1 + 3x_2 + 10x_3 &= 15.
\end{aligned}
\]
\end{enumerate}

\subsection*{Pseudocode}
\begin{enumerate}
\item Write pseudocode for the Thomas algorithm (forward elimination and back substitution).
\item Write pseudocode for one Gauss--Seidel iteration for a general $n \times n$ system.
\end{enumerate}

%------------------------------------------------
\section{Nonlinear Equations and Newton's Method}

\subsection*{Open-Ended Questions}
\begin{enumerate}
\item What does it mean to linearize a nonlinear equation?
\item Why is Newton's method considered a root-finding algorithm?
\item What conditions are required for quadratic convergence?
\item Why does Newton's method require solving a linear system at each iteration (for systems)?
\end{enumerate}

\subsection*{Hand Calculations}
\begin{enumerate}
\item Derive Newton's update formula for solving $f(x)=0$ using a Taylor series expansion.
\item Given $f(x) = x^2 - 2$, perform two Newton iterations starting from $x_0 = 1$.
\item For the nonlinear function
\[
f(\sigma) = \frac{\sigma}{E} + \alpha\left(\frac{\sigma}{\sigma_0}\right)^m - \varepsilon,
\]
identify $f'(\sigma)$ symbolically.
\end{enumerate}

\subsection*{Pseudocode}
\begin{enumerate}
\item Write pseudocode for Newton's method for a scalar equation.
\item Write pseudocode for Newton's method applied to a system of nonlinear equations using a Jacobian matrix.
\end{enumerate}

%------------------------------------------------
\section{Big Picture Questions}
\begin{enumerate}
\item Why do so many engineering problems reduce to solving $Ax=b$ or $F(x)=0$?
\item How does Newton's method for nonlinear systems mirror what we do in computational mechanics solvers?
\item Explain how symbolic computation (SymPy) and numerical computation (NumPy) complement each other.
\end{enumerate}

\bigskip
\noindent\textbf{Study Advice:}
Be prepared to explain ideas in words, carry out small calculations cleanly by hand, and write clear, logically structured pseudocode. The exam emphasizes understanding over memorization.

\end{document}