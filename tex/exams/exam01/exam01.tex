\documentclass[11pt]{article}

\usepackage{amsmath,amssymb}
\usepackage{geometry}
\usepackage{enumitem}
\usepackage{array}
\usepackage{fancyhdr}

\geometry{margin=1in}

\pagestyle{fancy}
\fancyhf{}
\lhead{MMAE 350 --- Computational Mechanics}
\rhead{Midterm Exam 1}
\cfoot{\thepage}

\setlength{\parskip}{0.75em}
\setlength{\parindent}{0pt}

\begin{document}

\begin{center}
    {\Large \textbf{Midterm Exam 1}}\\[0.5em]
    {\large MMAE 350 --- Computational Mechanics}\\[0.5em]
    \textbf{Time: 75 minutes}
\end{center}

\vspace{1em}

\textbf{Instructions}
\begin{itemize}
    \item Answer all questions.
    \item Show all work for calculation problems.
    \item Write clearly and concisely.
    \item One handwritten formula sheet (front/back) is allowed.
    \item No computers, tablets, or calculators with symbolic capability.
\end{itemize}

\vspace{1em}

\hrule
\vspace{1em}

\section*{Part I: Conceptual Understanding \hfill (30 points)}

\textbf{Question 1: Direct vs.\ Iterative Solvers \hfill (10 points)}

\begin{enumerate}[label=(\alph*)]
    \item In your own words, explain the difference between a \emph{direct solver} and an \emph{iterative solver} for linear systems.
    \item Give one advantage and one disadvantage of each type of solver.
    \item Why is \texttt{numpy.linalg.solve} typically faster than a loop-based Gauss--Seidel implementation written in Python?
\end{enumerate}

\vspace{1em}

\textbf{Question 2: Matrix Structure \hfill (10 points)}

A linear system arises from a one-dimensional finite difference discretization of a differential equation.

\begin{enumerate}[label=(\alph*)]
    \item What special structure does the resulting matrix usually have?
    \item How does this structure affect storage requirements and computational cost?
    \item Which solver introduced in this course is specifically designed to exploit this structure?
\end{enumerate}

\vspace{1em}

\textbf{Question 3: Residuals and Convergence \hfill (10 points)}

\begin{enumerate}[label=(\alph*)]
    \item Define the residual vector \( \mathbf{r} = \mathbf{b} - A\mathbf{x} \).
    \item What does the residual measure, mathematically or physically?
    \item Why is the \emph{relative residual norm} often preferred over the absolute residual norm?
\end{enumerate}

\vspace{1em}
\hrule
\vspace{1em}

\section*{Part II: Hand Calculations \hfill (30 points)}

\textbf{Question 4: One Gauss--Seidel Iteration \hfill (15 points)}

Consider the linear system
\[
\begin{bmatrix}
4 & -1 & 0 \\
-1 & 4 & -1 \\
0 & -1 & 3
\end{bmatrix}
\begin{bmatrix}
x_1 \\ x_2 \\ x_3
\end{bmatrix}
=
\begin{bmatrix}
15 \\ 10 \\ 10
\end{bmatrix}.
\]

\begin{enumerate}[label=(\alph*)]
    \item Starting from the initial guess \( \mathbf{x}^{(0)} = (0,0,0)^T \), perform \textbf{one Gauss--Seidel iteration}.
    \item Clearly indicate which values are updated and reused during the sweep.
\end{enumerate}

\vspace{1em}

\textbf{Question 5: Thomas Algorithm \hfill (15 points)}

Solve the following tridiagonal system using the Thomas algorithm:
\[
\begin{aligned}
2x_1 - x_2 &= 1, \\
-x_1 + 2x_2 - x_3 &= 0, \\
\phantom{-} -x_2 + 2x_3 &= 1.
\end{aligned}
\]

\begin{enumerate}[label=(\alph*)]
    \item Write down the modified coefficients after the forward sweep.
    \item Perform the backward substitution to compute \(x_1, x_2, x_3\).
\end{enumerate}

\vspace{1em}
\hrule
\vspace{1em}

\section*{Part III: Nonlinear Systems and Interpretation \hfill (30 points)}

\textbf{Question 6: Newton's Method (1D) \hfill (15 points)}

Consider the nonlinear equation
\[
f(x) = x^2 - 2.
\]

\begin{enumerate}[label=(\alph*)]
    \item Write down the Newton update formula for this problem.
    \item Starting from \(x^{(0)} = 1\), perform \textbf{one Newton iteration}.
    \item Briefly explain why Newton's method converges faster than the bisection method when it converges.
\end{enumerate}

\vspace{1em}

\textbf{Question 7: Interpreting Computational Results \hfill (15 points)}

A student applies three methods to solve a linear system of increasing size \(n\):

\vspace{0.5em}

\begin{center}
\begin{tabular}{|l|l|}
\hline
\textbf{Method} & \textbf{Observed Behavior} \\
\hline
\texttt{numpy.linalg.solve} & Fast for small--medium \(n\), slows for large \(n\) \\
Gauss--Seidel (Python loop) & Slow, many iterations \\
Thomas algorithm & Fast, scales approximately linearly \\
\hline
\end{tabular}
\end{center}

\vspace{0.5em}

\begin{enumerate}[label=(\alph*)]
    \item Which method scales best and why?
    \item Why does Gauss--Seidel perform poorly in pure Python even though the algorithm is simple?
    \item For very large \(n\), which method would you choose and why?
\end{enumerate}

\vspace{1em}

\hrule
\vspace{1em}

\textbf{Question 7: Interpreting Newton's Method \hfill (15 points)}

A student applies Newton's method to solve a nonlinear equation \( f(x) = 0 \) using
different initial guesses. The following behavior is observed:

\begin{itemize}
    \item Initial guess \(x^{(0)} = 0.1\): method converges slowly
    \item Initial guess \(x^{(0)} = 1.0\): method converges rapidly
    \item Initial guess \(x^{(0)} = 2.5\): method fails to converge
\end{itemize}

\begin{enumerate}[label=(\alph*)]
    \item Explain why Newton's method can converge at very different rates depending on the initial guess.
    \item What properties of the function \(f(x)\) and its derivative \(f'(x)\) influence convergence?
    \item Give one practical strategy an engineer might use to improve the robustness of Newton's method in practice.
\end{enumerate}

\textbf{End of Exam}

\end{document}