\documentclass[11pt]{article}

\usepackage{amsmath,amsfonts,amssymb}
\usepackage{bm}
\usepackage{geometry}
\usepackage{enumitem}
\usepackage{hyperref}

\geometry{margin=1in}

\title{\textbf{MMAE 350 -- Homework 3}\\
Direct and Iterative Solvers for Linear Systems}
\author{}
\date{Due: \textbf{Friday, February 7, 2026 at 11:59 PM}}

\begin{document}
\maketitle

\vspace{-1em}

\section*{Overview}
In this homework, you will explore two fundamental approaches for solving linear systems of equations:
\begin{itemize}
  \item the \emph{Thomas Algorithm}, a direct solver specialized for tridiagonal systems, and
  \item the \emph{Gauss--Seidel method}, an iterative solver applicable to more general sparse systems.
\end{itemize}

You will solve each problem analytically by hand and then verify your results using a Jupyter notebook.
The goal is to understand both the \emph{algorithmic structure} and the \emph{computational behavior} of each method.

\section*{Instructions}
\begin{itemize}
  \item Show all steps clearly in your handwritten or typed solution.
  \item A companion Jupyter notebook is provided to \emph{check} your results.
  \item The notebook should not be used to replace the required analytical work.
\end{itemize}

Submit the following to Canvas:
\begin{itemize}
  \item A single PDF containing your written solutions.
  \item Your completed Jupyter notebook (\texttt{.ipynb}).
\end{itemize}

%-------------------------------------------------
\section*{Problem 1: Thomas Algorithm (Direct Solver)}

Consider the tridiagonal linear system
\[
\bm{A}\bm{x} = \bm{b},
\]
where
\[
\bm{A} =
\begin{bmatrix}
4 & -1 & 0 & 0 \\
-1 & 4 & -1 & 0 \\
0 & -1 & 4 & -1 \\
0 & 0 & -1 & 3
\end{bmatrix},
\qquad
\bm{b} =
\begin{bmatrix}
15 \\ 10 \\ 10 \\ 10
\end{bmatrix}.
\]

\subsection*{(a) Hand solution using the Thomas Algorithm}
\begin{enumerate}[label=(\alph*)]
  \item Write the forward sweep equations for the Thomas Algorithm.
  \item Compute the modified coefficients step by step.
  \item Perform back substitution to obtain the solution vector $\bm{x}$.
\end{enumerate}

Clearly show your forward-sweep table and all intermediate values.

\subsection*{(b) Verification using a Jupyter notebook}
Using the provided notebook:
\begin{itemize}
  \item Enter the matrix $\bm{A}$ and vector $\bm{b}$.
  \item Solve the system using a Thomas Algorithm implementation.
  \item Confirm that the numerical solution matches your hand calculation.
\end{itemize}

%-------------------------------------------------
\section*{Problem 2: Gauss--Seidel Method (Iterative Solver)}

In this problem, you will solve the \emph{same linear system} using the Gauss--Seidel method.

\subsection*{(a) One Gauss--Seidel sweep by hand}
Assume the initial guess
\[
\bm{x}^{(0)} = \begin{bmatrix} 0 & 0 & 0 & 0 \end{bmatrix}^T.
\]

Perform one full Gauss--Seidel sweep to compute $\bm{x}^{(1)}$.
Clearly indicate which values are:
\begin{itemize}
  \item newly updated within the current sweep, and
  \item carried over from the previous iteration.
\end{itemize}

\subsection*{(b) Second sweep}
Perform a second Gauss--Seidel sweep to compute $\bm{x}^{(2)}$.
You may compute all components explicitly or compute one component in detail and explain the pattern.

\subsection*{(c) Convergence discussion}
In 2--3 sentences, answer the following:
\begin{itemize}
  \item Why does the Gauss--Seidel method converge for this system?
  \item How does the converged solution compare to the Thomas Algorithm solution?
\end{itemize}

\subsection*{(d) Verification using a Jupyter notebook}
Using the provided notebook:
\begin{itemize}
  \item Implement the Gauss--Seidel method.
  \item Track the solution vector and residual norm versus iteration.
  \item Verify convergence to the same solution obtained in Problem~1.
\end{itemize}

%-------------------------------------------------
\section*{Submission Checklist}
Before submitting, make sure that:
\begin{itemize}
  \item All analytical steps are clearly shown.
  \item Numerical values are reported with reasonable precision.
  \item Your notebook runs without errors.
  \item Your notebook output confirms your hand calculations.
\end{itemize}

\section*{Learning Takeaways}
After completing this assignment, you should be able to:
\begin{itemize}
  \item Apply a direct solver to a structured linear system.
  \item Perform Gauss--Seidel iterations by hand.
  \item Explain the difference between direct and iterative solvers.
  \item Use computational tools to verify analytical work.
\end{itemize}

\end{document}