\documentclass[11pt]{article}

\usepackage{amsmath}
\usepackage{tikz}
\usetikzlibrary{calc}

\usepackage{graphicx}
\usepackage{siunitx}
\usepackage{geometry}
\usepackage{enumitem}
\geometry{margin=1in}
\sisetup{detect-all}

\usepackage{pgfplots}
\pgfplotsset{compat=1.18}



\begin{document}



\section*{Notebook Example: 4-Bar Linkage Position Analysis (Nonlinear System)}
\label{sec:fourbar-newton}

A planar four-bar linkage consists of four rigid links connected by ideal
pin joints. The ground link has length $r_1$ and connects two fixed pivots
$O_2$ and $O_4$. The input crank has length $r_2$ and rotates about $O_2$
with a prescribed input angle $\theta_2$. The coupler has length $r_3$ and
connects joint $A$ to joint $B$. The output rocker has length $r_4$ and
rotates about $O_4$ with angle $\theta_4$.

\begin{figure}[h]
\centering
\begin{tikzpicture}[scale=1.05, line cap=round, line join=round]
  % --- Parameters for drawing only (not the analysis) ---
  \def\rOne{5.0}   % ground length (visual)
  \def\rTwo{2.3}   % input
  \def\rThree{4.0} % coupler
  \def\rFour{3.0}  % output
  \def\thTwo{25}   % degrees (given)
  \def\thFour{80}  % degrees (unknown in problem; set for picture)

  % Fixed pivots
  \coordinate (O2) at (0,0);
  \coordinate (O4) at (\rOne,0);

  % Joint A from input crank
  \coordinate (A) at ({\rTwo*cos(\thTwo)},{\rTwo*sin(\thTwo)});

  % Joint B from output rocker (for picture)
  \coordinate (B) at ({\rOne+\rFour*cos(\thFour)},{\rFour*sin(\thFour)});

  % Links
  \draw[thick] (O2) -- (O4) node[midway, below] {$r_1$};
  \draw[thick] (O2) -- (A)  node[midway, above]  {$r_2$};
  \draw[thick] (A)  -- (B)  node[midway, above] {$r_3$};
  \draw[thick] (O4) -- (B)  node[midway, right] {$r_4$};

  % Joints
  \fill (O2) circle (2pt) node[below left] {$O_2$};
  \fill (O4) circle (2pt) node[below right] {$O_4$};
  \fill (A)  circle (2pt) node[above left] {$A$};
  \fill (B)  circle (2pt) node[above right] {$B$};

  % Reference x-axis at O2 and O4
  \draw[->, thin] (O2) -- ++(1.3,0) node[below] {$x$};
  \draw[->, thin] (O2) -- ++(0,1.3) node[left] {$y$};

  % Angle arcs
  % theta2 at O2 (given)
  \draw[thin] (1.0,0) arc[start angle=0, end angle=\thTwo, radius=1.0];
  \node at (1.25,0.25) {$\theta_2$};

  % theta4 at O4 (unknown)
  \begin{scope}
    \coordinate (O4x) at ($(O4)+(1.0,0)$);
    \draw[thin] (O4x) arc[start angle=0, end angle=\thFour, radius=1.0];
    \node at ($(O4)+(1.2,0.45)$) {$\theta_4$};
  \end{scope}
  
  % draw local x-axis at A
      \draw[->, thin] (O4) -- ++(1.2,0);

  % theta3 at A (unknown): measure relative to +x direction through A
  \begin{scope}
    % compute direction of AB for the drawn configuration (for arc only)
    \path let
      \p1 = (A),
      \p2 = (B)
    in
      % draw local x-axis at A
      \draw[->, thin] (A) -- ++(1.0,0);
    ;
    % approximate theta3 arc using the AB direction angle in the picture
    % (angle value is conceptual in the figure; exact numeric not needed)
    \draw[thin] (A) ++(0.9,0) arc[start angle=0, end angle=30, radius=0.9];
    \node at ($(A)+(1.1,0.25)$) {$\theta_3$};
  \end{scope}

  % Notes
  \node[align=left] at ($(O2)+(2.7,-1.15)$)
    {\footnotesize Given: $r_1,r_2,r_3,r_4,\ \theta_2$\quad Solve for: $\theta_3,\theta_4$};

\end{tikzpicture}
\caption{Planar four-bar linkage.}
\label{fig:fourbar}
\end{figure}


For a given set of link lengths $(r_1,r_2,r_3,r_4)$ and a prescribed input
angle $\theta_2$, determine the configuration of the linkage by solving for
the unknown angles
\[
\theta_3 \quad \text{and} \quad \theta_4.
\]
This is a nonlinear system of equations because the geometry involves
$\sin(\cdot)$ and $\cos(\cdot)$ terms.

\subsection*{Coordinate setup}
Place the mechanism in the $xy$-plane with fixed pivots
\[
O_2=(0,0), \qquad O_4=(r_1,0).
\]
Let the joint locations be:
\[
A = O_2 + ( r_2\cos\theta_2,\ r_2\sin\theta_2),
\qquad
B = O_4 + ( r_4\cos\theta_4,\ r_4\sin\theta_4).
\]
The coupler constraint is that the distance between $A$ and $B$ equals $r_3$.

\subsection*{Loop-closure (nonlinear system)}
The vector loop-closure equation is
\[
\mathbf{r}_2 + \mathbf{r}_3 = \mathbf{r}_1 + \mathbf{r}_4,
\]
which, in components, yields the nonlinear system
\[
\begin{aligned}
f_1(\theta_3,\theta_4) &= r_2\cos\theta_2 + r_3\cos\theta_3 - r_1 - r_4\cos\theta_4 = 0,\\
f_2(\theta_3,\theta_4) &= r_2\sin\theta_2 + r_3\sin\theta_3 - r_4\sin\theta_4 = 0.
\end{aligned}
\]
Define
\[
F(\mathbf{x})=
\begin{bmatrix}
f_1(\theta_3,\theta_4)\\[2pt]
f_2(\theta_3,\theta_4)
\end{bmatrix},
\qquad
\mathbf{x}=
\begin{bmatrix}
\theta_3\\
\theta_4
\end{bmatrix}.
\]
We seek $\mathbf{x}$ such that $F(\mathbf{x})=\mathbf{0}$.


% ------------------------------------------------------------

\end{document}