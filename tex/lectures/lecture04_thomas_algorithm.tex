
\documentclass{beamer}

\usetheme{Madrid}
\usecolortheme{default}

\title{Thomas Algorithm for Tridiagonal Systems}
\author{MMAE 350}
\institute{Illinois Institute of Technology}
\date{}

\begin{document}

\begin{frame}
\titlepage
\end{frame}

\begin{frame}{Why the Thomas Algorithm?}
\begin{itemize}
\item Many engineering problems lead to \textbf{tridiagonal systems}
\item Examples:
\begin{itemize}
  \item 1D heat conduction (finite differences)
  \item 1D Poisson equation
  \item Beam and bar discretizations
\end{itemize}
\item Standard Gaussian elimination costs $\mathcal{O}(n^3)$
\item Tridiagonal structure allows a solver in $\mathcal{O}(n)$
\end{itemize}
\end{frame}

\begin{frame}{Tridiagonal Linear System}
We solve
\begin{equation}
A x = d
\end{equation}

where $\mathbf{A}$ is tridiagonal:
\begin{equation}
\mathbf{A} =
\begin{bmatrix}
b_0 & c_0 &        &        & 0 \\
a_1 & b_1 & c_1    &        &   \\
    & a_2 & b_2    & c_2    &   \\
    &     & \ddots & \ddots & c_{n-2} \\
0   &     &        & a_{n-1} & b_{n-1}
\end{bmatrix}
\end{equation}
\end{frame}

\begin{frame}{Key Idea}
\begin{itemize}
\item Thomas algorithm is Gaussian elimination \textbf{without fill-in}
\item We never store the full matrix --- only the three diagonals
\[
\{a_i\},\quad \{b_i\},\quad \{c_i\}
\]
\item Algorithm steps:
\begin{itemize}
\item Forward sweep (eliminate subdiagonal)
\item Back substitution (solve from bottom to top)
\end{itemize}
\end{itemize}
\end{frame}

% --------------------------------------------------------------------
% Added after Slide 4: 5x5 hand calculation that motivates the pattern
% --------------------------------------------------------------------

\begin{frame}{5$\times$5 Hand Calculation: A Concrete Example}
Consider the tridiagonal system $Ax=d$ with
\begin{equation}
A =
\begin{bmatrix}
4 & -1 & 0 & 0 & 0\\
-1 & 4 & -1 & 0 & 0\\
0 & -1 & 4 & -1 & 0\\
0 & 0 & -1 & 4 & -1\\
0 & 0 & 0 & -1 & 4
\end{bmatrix},
\qquad
d =
\begin{bmatrix}
3\\
2\\
2\\
2\\
3
\end{bmatrix}.
\end{equation}
\pause
\begin{itemize}
\item This is a common finite-difference matrix (Poisson / diffusion).
\item We will eliminate the subdiagonal entries one row at a time.
\end{itemize}
\end{frame}

\begin{frame}{5$\times$5 Hand Calculation: Eliminate $a_1$}
Row 1 (index 0) is
\[
4x_0 - x_1 = 3.
\]
Row 2 (index 1) is
\[
- x_0 + 4x_1 - x_2 = 2.
\]
To eliminate $x_0$ from Row 2, multiply Row 1 by
\[
m_1 = \frac{a_1}{b_0}=\frac{-1}{4},
\]
and subtract:
\[
\text{Row 2} \leftarrow \text{Row 2} - m_1(\text{Row 1}).
\]
\pause
The updated Row 2 becomes
\[
\underbrace{\left(4 - m_1(-1)\right)}_{b'_1}x_1 - x_2
=
\underbrace{\left(2 - m_1(3)\right)}_{d'_1}.
\]
\end{frame}

\begin{frame}{5$\times$5 Hand Calculation: The Pattern Emerges}
At the next step, we eliminate $a_2$ using the \emph{updated} Row 2.
This repeats with the same structure:
\begin{itemize}
\item a multiplier $m_i$ (one number)
\item an updated diagonal entry $b'_i$
\item an updated right-hand side entry $d'_i$
\end{itemize}
\pause
This leads directly to the forward-sweep recurrences on the next slide.
\end{frame}

\begin{frame}{Forward Sweep}
Initialization:
\begin{equation}
b'_0 = b_0, \qquad d'_0 = d_0
\end{equation}

For $i = 1, \dots, n-1$:
\begin{equation}
m_i = \frac{a_i}{b'_{i-1}}, \qquad
b'_i = b_i - m_i c_{i-1}, \qquad
d'_i = d_i - m_i d'_{i-1}.
\end{equation}
\end{frame}

\begin{frame}{Back Substitution}
After the forward sweep, the system is upper triangular.
\begin{equation}
x_{n-1} = \frac{d'_{n-1}}{b'_{n-1}}
\end{equation}

For $i = n-2, \dots, 0$:
\begin{equation}
x_i = \frac{d'_i - c_i x_{i+1}}{b'_i}
\end{equation}
\end{frame}

\begin{frame}{Summary}
\begin{itemize}
\item Specialized solver for tridiagonal systems
\item Linear complexity $\mathcal{O}(n)$ and storage $\mathcal{O}(n)$
\item Forward sweep updates $b'_i$ and $d'_i$ using one multiplier $m_i$
\item Back substitution recovers $x$ from bottom to top
\end{itemize}
\end{frame}

\begin{frame}{Why Order $n^3$ vs.\ Order $n$ Matters}

\begin{columns}
\column{0.55\textwidth}
\textbf{Computational cost scaling}

\begin{center}
\begin{tabular}{c c c}
\hline
Problem size $n$ & $n^3$ & $n$ \\
\hline
$10$ & $10^3$ & $10$ \\
$10^2$ & $10^6$ & $10^2$ \\
$10^3$ & $10^9$ & $10^3$ \\
$10^6$ & $10^{18}$ & $10^6$ \\
\hline
\end{tabular}
\end{center}

\column{0.45\textwidth}
\textbf{Interpretation}
\begin{itemize}
\item Dense Gaussian elimination:
\[
\mathcal{O}(n^3)
\]
\item Thomas algorithm:
\[
\mathcal{O}(n)
\]
\item Doubling $n$:
\begin{itemize}
  \item $n^3 \rightarrow 8\times$ more work
  \item $n \rightarrow 2\times$ more work
\end{itemize}
\end{itemize}

\end{columns}

\vspace{0.5em}
\begin{block}{Key takeaway}
\centering
\emph{Thomas is fast because it exploits structure, not because it changes the physics.}
\end{block}

\end{frame}


\end{document}
