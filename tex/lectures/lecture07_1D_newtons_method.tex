\documentclass[aspectratio=169]{beamer}

% -------------------------------------------------
% Packages
% -------------------------------------------------
\usepackage{amsmath, amsfonts}
\usepackage{graphicx}
\usepackage{booktabs}
\usepackage{hyperref}
\usepackage{bm}

% Figure search paths (relative to tex/lectures/)
\graphicspath{{../figures/lectures/}{../figures/shared/}}

% -------------------------------------------------
% Notes (speaker notes)
% -------------------------------------------------
\usepackage{pgfpages}
% Uncomment ONE of these for speaker notes:
% \setbeameroption{show notes} % notes only (for printing notes)
% \setbeameroption{show notes on second screen=right} % slides + notes

% -------------------------------------------------
% TikZ
% -------------------------------------------------
\usepackage{tikz}
\usetikzlibrary{matrix, calc}
\usepackage{xcolor} % (tikz loads xcolor, but explicit is fine)
\usetikzlibrary{matrix,fit,backgrounds}

% -------------------------------------------------
% Themes
%--------------------------------------------------
\usetheme{default}
\usecolortheme{default}
\setbeamertemplate{navigation symbols}{}


% put this macro in the preamble (or just above the frame)
%\newcommand{\mcell}[2]{\colorbox{#1}{\makebox[1.05cm][c]{\strut #2}}}


\newcommand{\newval}[1]{\textcolor{green!60!black}{#1}}
\newcommand{\oldval}[1]{\textcolor{orange!80!black}{#1}}
\newcommand{\diagval}[1]{\textcolor{blue!70!black}{#1}}  % adjust path if needed

\title{Newton's Method for Nonlinear Equations}
\subtitle{One-Dimensional Root Finding}
\author{MMAE 350 --- Computational Mechanics}
\date{}

\begin{document}

% -------------------------------------------------
\begin{frame}
\titlepage
\end{frame}

% -------------------------------------------------
\begin{frame}{Motivation}
\begin{itemize}
\item Many engineering problems lead to equations of the form
\[
f(x) = 0
\]
\item These equations are often \emph{nonlinear}
\item Closed-form solutions rarely exist
\item We need a \emph{systematic numerical method}
\end{itemize}

\vspace{1em}
\pause
\textbf{Newton's method} is one of the most important tools in computational mechanics.
\end{frame}

% -------------------------------------------------
\begin{frame}{Examples from Mechanics}
Nonlinear equations arise when:
\begin{itemize}
\item Material behavior is nonlinear (plasticity, large strain)
\item Geometry changes with deformation
\item Loads depend on displacement
\end{itemize}

\vspace{1em}
Examples:
\begin{itemize}
\item Nonlinear stress--strain laws
\item Contact problems
\item Large deformation equilibrium
\end{itemize}
\end{frame}

% -------------------------------------------------
\begin{frame}{The Root-Finding Problem}
We want to solve:
\[
f(x) = 0
\]

\begin{itemize}
\item $f(x)$ is known
\item $x$ is unknown
\item We want the value of $x$ where the curve crosses zero
\end{itemize}

\vspace{1em}
\pause
This is called a \emph{root} of the function.
\end{frame}

% -------------------------------------------------
\begin{frame}{Geometric Idea Behind Newton's Method}
Start with an initial guess $x^{(0)}$.

\pause
\begin{itemize}
\item Approximate $f(x)$ near $x^{(k)}$ by its tangent line
\item Use the tangent line to predict where $f(x)=0$
\item Use that prediction as the next guess
\end{itemize}

\vspace{1em}
\pause
Newton's method is \emph{repeated linearization}.
\end{frame}

% -------------------------------------------------
\begin{frame}{Linearization via Taylor Series}
Expand $f(x)$ about the current iterate $x^{(k)}$:
\[
f(x) \approx f(x^{(k)}) + f'(x^{(k)})(x - x^{(k)})
\]

Set the approximation equal to zero:
\[
0 \approx f(x^{(k)}) + f'(x^{(k)})(x^{(k+1)} - x^{(k)})
\]
\end{frame}

% -------------------------------------------------
\begin{frame}{Newton Update Formula}
Solving for $x^{(k+1)}$ gives:
\[
\boxed{
x^{(k+1)} = x^{(k)} - \frac{f(x^{(k)})}{f'(x^{(k)})}
}
\]

\pause
This is the \textbf{Newton update}.
\end{frame}

% -------------------------------------------------
\begin{frame}{Algorithm Summary}
\textbf{Newton's Method (1D)}

\begin{enumerate}
\item Choose an initial guess $x^{(0)}$
\item For $k = 0,1,2,\dots$:
\begin{itemize}
\item Evaluate $f(x^{(k)})$
\item Evaluate $f'(x^{(k)})$
\item Update:
\[
x^{(k+1)} = x^{(k)} - \frac{f(x^{(k)})}{f'(x^{(k)})}
\]
\end{itemize}
\item Stop when $|f(x^{(k)})|$ is small
\end{enumerate}
\end{frame}

% -------------------------------------------------
\begin{frame}{Convergence Behavior}
If Newton's method converges:
\begin{itemize}
\item Convergence is typically \textbf{quadratic}
\item Error decreases very rapidly near the root
\end{itemize}

\vspace{1em}
\pause
But:
\begin{itemize}
\item Requires a good initial guess
\item Can fail if $f'(x)$ is small or zero
\end{itemize}
\end{frame}

% -------------------------------------------------
\begin{frame}{Why Initial Guess Matters}
Newton's method is \emph{local}.

\begin{itemize}
\item Works best when starting near the solution
\item Poor guesses can lead to divergence
\item In mechanics, physics often gives a good initial guess
\end{itemize}

\vspace{1em}
\pause
\textbf{Example:} Linear elasticity is often a good starting point.
\end{frame}

% -------------------------------------------------
\begin{frame}{Newton's Method in Mechanics Language}
\begin{itemize}
\item $f(x)$ = residual (force imbalance)
\item $f'(x)$ = tangent stiffness
\item Newton update = solve linearized equilibrium
\end{itemize}

\vspace{1em}
\pause
Each Newton step:
\begin{itemize}
\item Linearizes the problem
\item Solves a linear system
\item Updates the configuration
\end{itemize}
\end{frame}

% -------------------------------------------------
\begin{frame}{Connection to Finite Elements}
In 1D:
\[
f'(x) = \frac{df}{dx}
\]

In many degrees of freedom:
\[
\mathbf{K}_T \Delta \mathbf{u} = -\mathbf{R}
\]

\pause
\begin{itemize}
\item Tangent stiffness matrix
\item Residual force vector
\item Same idea, larger system
\end{itemize}
\end{frame}

% -------------------------------------------------
\begin{frame}{Stopping Criteria}
Common choices:
\begin{itemize}
\item $|f(x^{(k)})| < \text{tolerance}$
\item $|x^{(k+1)} - x^{(k)}| < \text{tolerance}$
\end{itemize}

\vspace{1em}
\pause
In mechanics, residual-based criteria are most common.
\end{frame}

% -------------------------------------------------
\begin{frame}{What Can Go Wrong?}
\begin{itemize}
\item Poor initial guess
\item Very stiff or very soft nonlinearities
\item Large step overshooting the solution
\end{itemize}

\vspace{1em}
\pause
Fixes:
\begin{itemize}
\item Damping (line search)
\item Smaller load steps
\item Better initial guesses
\end{itemize}
\end{frame}

% -------------------------------------------------
\begin{frame}{Takeaway}
\begin{itemize}
\item Newton's method solves nonlinear equations by repeated linearization
\item It is fast and powerful when used correctly
\item It is the foundation of nonlinear finite element analysis
\end{itemize}

\vspace{1em}
\pause
\textbf{In this course:} Newton’s method is not just an algorithm — it is a modeling idea.
\end{frame}

\end{document}