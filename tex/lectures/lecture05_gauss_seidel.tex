
\documentclass{beamer}

% --- Theme / packages ---
% -------------------------------------------------
% Packages
% -------------------------------------------------
\usepackage{amsmath, amsfonts}
\usepackage{graphicx}
\usepackage{booktabs}
\usepackage{hyperref}
\usepackage{bm}

% Figure search paths (relative to tex/lectures/)
\graphicspath{{../figures/lectures/}{../figures/shared/}}

% -------------------------------------------------
% Notes (speaker notes)
% -------------------------------------------------
\usepackage{pgfpages}
% Uncomment ONE of these for speaker notes:
% \setbeameroption{show notes} % notes only (for printing notes)
% \setbeameroption{show notes on second screen=right} % slides + notes

% -------------------------------------------------
% TikZ
% -------------------------------------------------
\usepackage{tikz}
\usetikzlibrary{matrix, calc}
\usepackage{xcolor} % (tikz loads xcolor, but explicit is fine)
\usetikzlibrary{matrix,fit,backgrounds}

% -------------------------------------------------
% Themes
%--------------------------------------------------
\usetheme{default}
\usecolortheme{default}
\setbeamertemplate{navigation symbols}{}


% put this macro in the preamble (or just above the frame)
%\newcommand{\mcell}[2]{\colorbox{#1}{\makebox[1.05cm][c]{\strut #2}}}


\newcommand{\newval}[1]{\textcolor{green!60!black}{#1}}
\newcommand{\oldval}[1]{\textcolor{orange!80!black}{#1}}
\newcommand{\diagval}[1]{\textcolor{blue!70!black}{#1}}


\title{Gauss--Seidel Iteration for Linear Systems}
\subtitle{Direct vs.\ Iterative Solvers}
\author{MMAE 350: Computational Mechanics}
\date{}

\begin{document}

%-------------------------------------------------
\begin{frame}
  \titlepage
\end{frame}

%-------------------------------------------------
\begin{frame}{Motivation}
We have already seen the \emph{Thomas Algorithm}:
\begin{itemize}
  \item Direct
  \item Exact
  \item $\mathcal{O}(n)$ for tridiagonal systems
\end{itemize}

\vspace{0.3cm}

What if:
\begin{itemize}
  \item The matrix is not tridiagonal?
  \item The system is very large?
  \item We want an approximate solution that improves iteratively?
\end{itemize}
\end{frame}

%-------------------------------------------------
\begin{frame}{Problem Statement}
We seek to solve the linear system
\[
\bm{A}\bm{x} = \bm{b},
\]
where:
\begin{itemize}
  \item $\bm{A} \in \mathbb{R}^{n \times n}$
  \item $\bm{x}, \bm{b} \in \mathbb{R}^n$
\end{itemize}

\vspace{0.3cm}

The Gauss--Seidel method constructs a sequence
\[
\bm{x}^{(0)}, \bm{x}^{(1)}, \bm{x}^{(2)}, \ldots
\]
that converges to the solution.
\end{frame}

%-------------------------------------------------


%-------------------------------------------------
\begin{frame}{Component-Wise Update Formula}
For each equation $i = 1,\dots,n$:
\[
x_i^{(k+1)}
=
\frac{1}{a_{ii}}
\left(
b_i
-
\sum_{j<i} a_{ij} x_j^{(k+1)}
-
\sum_{j>i} a_{ij} x_j^{(k)}
\right).
\]

\vspace{0.3cm}

Key observation:
\begin{itemize}
  \item New values are used immediately
  \item Old values are used where necessary
\end{itemize}
\end{frame}

%-------------------------------------------------
\begin{frame}{Algorithm Outline}
\begin{enumerate}
  \item Choose an initial guess $\bm{x}^{(0)}$
  \item For $k = 0,1,2,\dots$:
  \begin{itemize}
    \item Loop through equations $i = 1$ to $n$
    \item Update $x_i^{(k+1)}$ using latest values
  \end{itemize}
  \item Check convergence
\end{enumerate}

\vspace{0.3cm}

One full sweep through the equations constitutes one iteration.
\end{frame}

%-------------------------------------------------
\begin{frame}{Convergence Behavior}
Gauss--Seidel converges if:
\begin{itemize}
  \item $\bm{A}$ is strictly diagonally dominant, or
  \item $\bm{A}$ is symmetric positive definite
\end{itemize}

\vspace{0.3cm}

Physical interpretation:
\begin{itemize}
  \item Diagonal dominance means local effects outweigh coupling
  \item Common in diffusion and elasticity problems
\end{itemize}
\end{frame}

%-------------------------------------------------
\begin{frame}{Gauss--Seidel vs.\ Jacobi}
\begin{center}
\begin{tabular}{lcc}
\textbf{Feature} & \textbf{Jacobi} & \textbf{Gauss--Seidel} \\
\hline
Uses updated values & No & Yes \\
Convergence speed & Slower & Faster \\
Parallel friendly & Yes & Less so \\
\end{tabular}
\end{center}

\vspace{0.3cm}

Gauss--Seidel can be viewed as \emph{Jacobi with memory}.
\end{frame}

%-------------------------------------------------
\begin{frame}{Gauss--Seidel vs.\ Thomas Algorithm}
\begin{center}
\begin{tabular}{lcc}
\textbf{Feature} & \textbf{Thomas} & \textbf{Gauss--Seidel} \\
\hline
Method type & Direct & Iterative \\
Accuracy & Exact & Approximate \\
Matrix type & Tridiagonal & General sparse \\
Cost & $\mathcal{O}(n)$ & $\mathcal{O}(n\,k)$ \\
\end{tabular}
\end{center}

\vspace{0.3cm}

Thomas is unbeatable when applicable;  
Gauss--Seidel works when Thomas cannot.
\end{frame}

%-------------------------------------------------
\begin{frame}{Role in Computational Mechanics}
Gauss--Seidel is foundational to:
\begin{itemize}
  \item Successive Over-Relaxation (SOR)
  \item Multigrid smoothers
  \item Nonlinear Newton solvers
\end{itemize}

\vspace{0.3cm}

Iterative thinking is essential for large-scale simulations.
\end{frame}

%-------------------------------------------------
\begin{frame}{Key Takeaways}
\begin{itemize}
  \item Gauss--Seidel is an \emph{iterative} solver
  \item Updates occur \emph{in-place}
  \item Convergence depends on matrix structure
  \item Complements direct methods like Thomas
\end{itemize}
\end{frame}
%======================================
% put this macro in the preamble (or just above the frame)
\newcommand{\mcell}[2]{\colorbox{#1}{\makebox[1.05cm][c]{\strut #2}}}

\begin{frame}{Gauss--Seidel Sweep: What is ``new'' vs ``old''?}
\centering
\small

% legend
\mcell{green!20}{\phantom{$a$}}\, already updated ($x^{(k+1)}$)\qquad
\mcell{orange!25}{\phantom{$a$}}\, not yet updated ($x^{(k)}$)\qquad
\mcell{blue!20}{\phantom{$a$}}\, diagonal / pivot

\vspace{0.5cm}

% ------- Overlay 1 -------
\only<1>{
\begin{tabular}{cccc}
\mcell{blue!20}{$a_{11}$} & \mcell{orange!25}{$a_{12}$} & \mcell{orange!25}{$a_{13}$} & \mcell{orange!25}{$a_{14}$}\\
\mcell{white}{$a_{21}$} & \mcell{white}{$a_{22}$} & \mcell{white}{$a_{23}$} & \mcell{white}{$a_{24}$}\\
\mcell{white}{$a_{31}$} & \mcell{white}{$a_{32}$} & \mcell{white}{$a_{33}$} & \mcell{white}{$a_{34}$}\\
\mcell{white}{$a_{41}$} & \mcell{white}{$a_{42}$} & \mcell{white}{$a_{43}$} & \mcell{white}{$a_{44}$}\\
\end{tabular}

\vspace{0.25cm}

\[
x_1^{(k+1)} =
\frac{1}{\diagval{a_{11}}}
\left(
b_1
-
\oldval{a_{12} x_2^{(k)}}
-
\oldval{a_{13} x_3^{(k)}}
-
\oldval{a_{14} x_4^{(k)}}
\right)
\]

\vspace{0.15cm}
Updating $x_1$: everything to the right uses old values.
}

% ------- Overlay 2 -------
\only<2>{
\begin{tabular}{cccc}
\mcell{white}{$a_{11}$} & \mcell{white}{$a_{12}$} & \mcell{white}{$a_{13}$} & \mcell{white}{$a_{14}$}\\
\mcell{green!20}{$a_{21}$} & \mcell{blue!20}{$a_{22}$} & \mcell{orange!25}{$a_{23}$} & \mcell{orange!25}{$a_{24}$}\\
\mcell{white}{$a_{31}$} & \mcell{white}{$a_{32}$} & \mcell{white}{$a_{33}$} & \mcell{white}{$a_{34}$}\\
\mcell{white}{$a_{41}$} & \mcell{white}{$a_{42}$} & \mcell{white}{$a_{43}$} & \mcell{white}{$a_{44}$}\\
\end{tabular}

\only<2>{
\[
x_2^{(k+1)} =
\frac{1}{\diagval{a_{22}}}
\left(
b_2
-
\newval{a_{21} x_1^{(k+1)}}
-
\oldval{a_{23} x_3^{(k)}}
-
\oldval{a_{24} x_4^{(k)}}
\right)
\]
}



\vspace{0.25cm}
Updating $x_2$: left is new (green), right is old (orange).
}

% ------- Overlay 3 -------
\only<3>{
\begin{tabular}{cccc}
\mcell{white}{$a_{11}$} & \mcell{white}{$a_{12}$} & \mcell{white}{$a_{13}$} & \mcell{white}{$a_{14}$}\\
\mcell{white}{$a_{21}$} & \mcell{white}{$a_{22}$} & \mcell{white}{$a_{23}$} & \mcell{white}{$a_{24}$}\\
\mcell{green!20}{$a_{31}$} & \mcell{green!20}{$a_{32}$} & \mcell{blue!20}{$a_{33}$} & \mcell{orange!25}{$a_{34}$}\\
\mcell{white}{$a_{41}$} & \mcell{white}{$a_{42}$} & \mcell{white}{$a_{43}$} & \mcell{white}{$a_{44}$}\\
\end{tabular}

\only<3>{
\[
x_3^{(k+1)} =
\frac{1}{\diagval{a_{33}}}
\left(
b_3
-
\newval{a_{31} x_1^{(k+1)}}
-
\newval{a_{32} x_2^{(k+1)}}
-
\oldval{a_{34} x_4^{(k)}}
\right)
\]
}

\vspace{0.25cm}
Updating $x_3$: $x_1,x_2$ are new; $x_4$ is old.
}

% ------- Overlay 4 -------
\only<4>{
\begin{tabular}{cccc}
\mcell{white}{$a_{11}$} & \mcell{white}{$a_{12}$} & \mcell{white}{$a_{13}$} & \mcell{white}{$a_{14}$}\\
\mcell{white}{$a_{21}$} & \mcell{white}{$a_{22}$} & \mcell{white}{$a_{23}$} & \mcell{white}{$a_{24}$}\\
\mcell{white}{$a_{31}$} & \mcell{white}{$a_{32}$} & \mcell{white}{$a_{33}$} & \mcell{white}{$a_{34}$}\\
\mcell{green!20}{$a_{41}$} & \mcell{green!20}{$a_{42}$} & \mcell{green!20}{$a_{43}$} & \mcell{blue!20}{$a_{44}$}\\
\end{tabular}

\only<4>{
\[
x_4^{(k+1)} =
\frac{1}{\diagval{a_{44}}}
\left(
b_4
-
\newval{a_{41} x_1^{(k+1)}}
-
\newval{a_{42} x_2^{(k+1)}}
-
\newval{a_{43} x_3^{(k+1)}}
\right)
\]
}

\vspace{0.25cm}
Updating $x_4$: all previous are new; diagonal is the pivot.
}

\end{frame}

%-------------------------------------------------
\begin{frame}{Convergence of Gauss--Seidel}
Gauss--Seidel produces a sequence
\[
\bm{x}^{(0)}, \bm{x}^{(1)}, \bm{x}^{(2)}, \ldots
\]
that (hopefully) converges to the solution of
\[
\bm{A}\bm{x} = \bm{b}.
\]

\vspace{0.3cm}

\begin{block}{Residual-Based Convergence (Preferred)}
Define the residual
\[
\bm{r}^{(k)} = \bm{b} - \bm{A}\bm{x}^{(k)}.
\]
Stop the iteration when
\[
\frac{\|\bm{r}^{(k)}\|}{\|\bm{b}\|} < \varepsilon.
\]
\end{block}

\vspace{0.2cm}

\begin{block}{Update-Based Convergence (Common in Practice)}
Alternatively, stop when the solution stops changing:
\[
\frac{\|\bm{x}^{(k+1)} - \bm{x}^{(k)}\|}{\|\bm{x}^{(k+1)}\|} < \varepsilon.
\]
\end{block}

\vspace{0.2cm}

\begin{itemize}
  \item Residual $\rightarrow$ equations are satisfied
  \item Update size $\rightarrow$ iteration has stabilized
\end{itemize}
\end{frame}

\end{document}
